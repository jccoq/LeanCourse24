\documentclass[a4paper,12pt]{article}

% Packages
\usepackage[utf8]{inputenc}    % Input encoding
\usepackage[T1]{fontenc}       % Font encoding
\usepackage{amsmath}           % Math symbols
\usepackage{amsfonts}          % Math fonts
\usepackage{amssymb}           % Extra symbols
\usepackage{geometry}          % Page geometry
\geometry{a4paper, margin=1in} % Set page size and margins
\usepackage{graphicx}          % Include graphics
\usepackage{hyperref}          % Hyperlinks in the document
\usepackage{amsthm}

\newtheorem{theorem}{Theorem}
\newtheorem{lemma}{Lemma}
% Title and Author Information
\title{Winding Number of a Curve - LEAN4 Project (Logbook)}
\author{Jorge Carrasco Coquillat \and
Juan Antonio Montalbán Vidal}
\date{\today} % Use \date{} to leave the date empty or set a custom date

\begin{document}

% Title Page
\maketitle

% Abstract
\begin{abstract}
In this logbook we aim to show how the journey with this project has been since day 0.
The content is divided in weeks, so we explain what we have been doing weekly,
decisions made and main difficulties we have had to tackle.
\end{abstract}

% Sections
\section{Choosing our project topic}
There were various options that seemed good for us, from Carmichael numbers
to orientability of manifolds. Lastly, we decided to work in a complex analysis topic:
the winding number.

The winding number of a curve can be defined in several ways, although we only
focused on the topological and analytic ones. Our main goal was going to be
to prove the equivalence between these definitions, albeit we quickly noticed
it was going to be a very stretch goal due to its complexity and the time we had.

Since this project focuses on the definition and properties of the winding number,
we will now give the definition of \textit{curve} that we have used - there is no a general
consensus on how to define them, especially regarding the definition interval.

For us, a curve $\gamma$ will be a $\cal{C}^1 (\text{I}, \mathbb{C})$ function, where I stands for the unit interval.

\begin{itemize}
  \item \textbf{Topological definition:} it uses the path-lifting property [ref]. Given
  a curve $\gamma : [0, 1] \to \mathbb{C}$ and $z_0 \in C$ satisfying $z \notin \text{im} \gamma$, the winding
  number of $\gamma$ around $z_0$ is defined as
  $$\omega (\gamma, z_0) = s(1) - s(0),$$
  where $(\rho, s)$ is the path written in polar coordinates, this is, the lifted path
  through the covering map
  $$p : \mathbb{R}_+ \times \mathbb{R} \to \mathbb{C}$$
  $$(\rho_0, s_0) \mapsto z_0 + \rho_0 e^{i2\pi s_0}.$$
  \item \textbf{Analytic definition:} The analytic definition is more straightforward. Consider the curve $\gamma$
  and the point $z_0$ as previously specified. The winding of $\gamma$ around $z_0$ is defined as
  $$\omega(\gamma, z_0) = \frac{1}{2\pi i} \int_{0}^{1} {\frac{\gamma'(t)}{\gamma(t)-z_0} \text{dt}}.$$
\end{itemize}

After spending half a week discussing wether we should try proving the equivalence of definitions we
noticed that there was a big problem regarding this issue: there are a lot of results missing in Mathlib
- especially from algebraic topology - which be would need just to get to define the winding number.

\section{Main formalized results}
The very first thing we had to tackle, far from a lemma or theorem, was giving the right - this is, the
most accurate one - definition of a curve. First we opted for defining it for an arbitrary interval $(a, b)$
but we rapidly noticed this was unnecesary and the vast majority of the literature simply used the unit interval,
so we decided to change to that.

But that was not the unique thing we had to change about the structure definition of a (closed) curve. At the beginning, we opted
for putting the continuity and differentiability conditions separately but, after hearing the given advices, we noticed
the best option was to put them together using the \verb|ContDiffOn| condition from Mathlib library.

\noindent \textbf{Remark 1:} we have also defined the concept of \textit{n-piecewise curve}, i.e., a curve that is the result
of the concatenation of n simple curves. This is a more general concept than a curve itself and, even though
we do not formalize theorems using them, we have thought it would be of interest to get them defined.

\noindent \textbf{Remark 2:} the sign of the winding number depends on the orientation we choose for going along
the curve or, in other words, if the curve winds \textit{clockwise} or \textit{counterclockwise} around the point.
For the sake of simplicity, we have chosen the counterclockwise orientation as the default one. In the case a curve is given
in the opposite orientation, we can always perform a change of basis to get back to desired scenario.

The main results we have formalized are the following:
\begin{theorem}
    Given a curve $\gamma$ and a point $z_0$ satisfying (P), then
    $$\omega(\gamma, z_0) \in \mathbb{Z}.$$
    In other words, the winding number of a curve around a point is always an integer.
\end{theorem}

\begin{theorem}
  The winding number $\omega(\gamma, z)$, seen as a function of $z$, is continuous in $\mathbb{C}\setminus \text{im}\gamma$.
\end{theorem}
\section{Main difficulties}
Here is the main content of the article.

\subsection{Difficulties and, in some cases, solutions}
DISCUSS!!!

\section{Conclusion and possible future work}
DISCUSS!!!

% References
\begin{thebibliography}{9}
\bibitem{sample1} Author Name, \textit{Book Title}, Publisher, Year.
\bibitem{sample2} Another Author, \textit{Another Book}, Another Publisher, Year.
\end{thebibliography}

\end{document}
